\title{
	\mbox{\Huge Base de Datos}\\
	%\mbox{\Huge Trabajo Pr�ctico Final}\\
	\mbox{}\\
	\mbox{\it Implementaci�n de algoritmo de ARIES }\\
	\mbox{\it sobre desarrollo de DBMS}\\
	\mbox{}\\
	Departamento de Computaci�n \\
	Facultad de Ciencias Exactas y Naturales\\
	Universidad de Buenos Aires
\date{}
}
\maketitle
\begin{center}
	\begin{tabular}{cc}
		Luciano Leggieri&Julian Berl�n\\
		{\small \verb"lleggieri@dc.uba.ar"}&{\small \verb"jberlin@dc.uba.ar"}\\
		\cr
		Victor Cabas&Facundo Pippia\\
		{\small \verb"vcabas@dc.uba.ar"}&{\small \verb"facundomensajes@hotmail.com"}\\
	\end{tabular}
	\vskip 5 mm
	Director:\\
	Alejandro Eidelsztein\\
	{\small \verb"ae0n@dc.uba.ar"}\\
\end{center}

{\centering {\bf Abstract}\\}
ARIES es un algoritmo de recuperaci�n muy popular que usa un enfoque steal / no-force. Comparado con otros esquemas de recuperaci�n, es simple y soporta diferentes grados de granularidad en los bloqueos. Luego de una ca�da, este algoritmo procede en tres fases para dejar la base de datos en el estado que tenia antes del crash. El objetivo de este trabajo es realizar una implementaci�n de ARIES en un motor de base de datos construido �ntegramente en Java. 

\vskip 5 mm

\texttt{Keywords: Bases de datos relacionales, ARIES, Transacciones, Recovery Manager}