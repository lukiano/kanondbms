\newpage
\section{Conclusiones}

Entre las cosas m�s destacables de haber realizado este trabajo fue la gran cantidad de cosas que pudimos aprender durante el transcurso del proyecto sobre lo que es un RDBMS y los aspectos relacionados con la recuperaci�n a un estado consistente luego de una ca�da.\\

Llegamos a la conclusi�n de que la familia de algoritmos de ARIES para la recuperaci�n y bloqueo es un m�todo muy ingenioso y eficaz que es muy usado actualmente en la industria por muchos de los motores relacionales de uso comercial y tambi�n otros para uso acad�mico. Este m�todo encara con precisi�n todos los problemas reales de concurrencia y recuperaci�n que est�n presentes en los RDBMS actuales que se usan �mbitos industriales.\\

Aprendimos de un sistema como ARIES, lo implementamos y nos dimos cuenta del buen funcionamiento y que tan complejo puede llegar a ser el manejo de los problemas de concurrencia y recuperaci�n. En un principio cre�mos que este m�todo s�lo era de aplicaci�n acad�mica pero luego vimos que era ampliamente utilizado y en varios casos extendido o personalizado de maneras interesantes (como paralelizaci�n de la recuperaci�n o uso de versionado en las p�ginas).\\

Este trabajo va servir de ayuda para comprender y observar el funcionamiento de un RDBMS ya que fue pensado con fines acad�micos reflejando de la forma m�s sencilla posible el funcionamiento interno del motor hacia el usuario. Mostramos como se implementa, desde cero, ARIES con un modelo acad�mico simple, modular, orientado a objetos, donde quien est� interesado en el tema puede ver el funcionamiento de ARIES en su totalidad para una mayor comprensi�n.\\

La metodolog�a que plantea el paper de ARIES se acopla bien con temas complejos como anidamiento de transacciones, recuperaci�n de ca�das o savepoints.\\





