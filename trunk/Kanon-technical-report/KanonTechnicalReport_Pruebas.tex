\newpage
\section{Pruebas sobre el sistema}

El objetivo de las pruebas fue principalmente verificar la correctitud
en toda la funcionalidad soportada por el servidor y el cliente.\\

A continuaci�n, ponemos en detalle una de la pruebas realizadas para
observar como se va modificando el log a medida que se ejecutan las
sentencias.\\

\subsection{Pruebas de funcionamiento}

\textbf{Prueba de inserci�n:}

La siguiente prueba muestra el resultado de ejecutar un insert sobre una
tabla existente:\\

\texttt{insert into proveedores values('Sancor', 'Cordoba 5400', 46532456);}\\

El Log registra los siguientes cambios:\\
   \begin{tabular}{|l|l|l|l|l|}
        \hline
         3976&INSERT&IdTx: 4&PrevLSN:0&\\
        \hline
				 &proveedores&Pag: 0&Reg: 0&\\
        \hline
         &&Col: 0&&Sancor\\
        \hline
  			 &&Col: 1&&Cordoba 5400\\
        \hline
  			 &&Col: 2&&46532456\\
        \hline
         4472&PREPARE&IdTx: 4&PrevLSN:4356&\\
        \hline
         &Lock:&proveedores&Pag: 0&Reg: 0\\
        \hline
         4540&END&IdTx: 4&PrevLSN:4472&\\
        \hline
    \end{tabular}\\

El primer evento corresponde a la inserci�n de un registro a la tabla
proveedores de la transacci�n 4, con LSN 3976 crea el registro 0 en la
pagina 0. Tambi�n se muestra en el log el contenido en cada columna.\\

Luego se hace un insert de registros de �ndices por cada columna, los
cuales no se visualizan en el log. Por eso el LSN 4472 tiene
PrevLSN:4356 y no 3976.\\

El evento PREPARE, indica el commit de la transacci�n y muestra los
locks obtenidos por la misma.\\

El evento END, muestra el cierre de la transacci�n.\\

El resultado de ejecutar un select sobre la tabla proveedores muestra lo
siguiente:\\

    \begin{tabular}{|l|c|l|}
        \hline
        \textbf{nombre} & \textbf{direccion} & \textbf{telefono} \\
        \hline
        Sancor&Cordoba 5400&46532456\\
        \hline
    \end{tabular}\\

\textbf{Prueba de rollback:}

Muestra el resultado de ejecutar una transacci�n explicita en la cual
se realiza la inserci�n de dos registros sobre la tabla proveedores y
luego se hace un rollback.\\
\texttt{begin transaction;\\
insert into proveedores values('Parmalat', 'Corrientes 2400',
46843221);\\
insert into proveedores values('La Serenisima', 'Callao 1030',
44327000);\\
rollback transaction;\\}

El log registra los siguientes cambios:\\
   \begin{tabular}{|l|l|l|l|l|}
        \hline
         4604&INSERT&IdTx: 6&PrevLSN:0&\\
        \hline
				 &proveedores&Pag: 0&Reg: 1&\\
        \hline
         &&Col: 0&&Parmalat\\
        \hline
  			 &&Col: 1&&Corrientes 2400\\
        \hline
  			 &&Col: 2&&46843221\\
        \hline
         5100&INSERT&IdTx: 6&PrevLSN:4984&\\
        \hline
         &proveedores&Pag: 0&Reg: 2&\\
        \hline
         &&Col: 0&&La Serenisima\\
        \hline
         &&Col: 1&&Callao 1030\\
        \hline
         &&Col: 2&&44327000\\
        \hline
         5596&ROLLBACK&IdTx: 6&PrevLSN:5480&\\
        \hline
         Lectura: 5596&ROLLBACK&proxLSN:5616&&\\
        \hline
         Lectura: 5480&INSERT\_INDEX&proxLSN:5596&&\\
        \hline
         Lectura: 5364&INSERT\_INDEX&proxLSN:5480&&\\
        \hline
         Lectura: 5248&INSERT\_INDEX&proxLSN:5364&&\\
        \hline
         Lectura: 5100&INSERT&proxLSN:5248&&\\
        \hline
         5868&CLR\_INSERT&IdTx: 6&PrevLSN:5784&\\
        \hline
         &&4984&&\\
        \hline
         &proveedores&Pag: 0&Reg: 2&\\
        \hline
         Lectura: 4984&INSERT\_INDEX&proxLSN:5100&&\\
        \hline
         Lectura: 4868&INSERT\_INDEX&proxLSN:4984&&\\
        \hline
         Lectura: 4752&INSERT\_INDEX&proxLSN:4868&&\\
        \hline
         Lectura: 4604&INSERT&proxLSN:4752&&\\
        \hline
         6196&CLR\_INSERT&IdTx: 6&PrevLSN:6112&\\
        \hline
         &&0&&\\
        \hline
         &proveedores&Pag: 0&Reg: 1&\\
        \hline
         6272&END&IdTx: 6&PrevLSN:6196&\\
        \hline
    \end{tabular}\\

La operaci�n begin inicializa el PrevLSN en 0. Los dos primeros
eventos corresponden a la inserci�n de dos registros en la tabla
proveedores.Luego se muestra el evento de ROLLBACK de la transacci�n 6
y se empieza a recorrer la misma leyendo el log desde la �ltima
operaci�n hasta la primera (aqu� se muestra los eventos sobre los
�ndices).\\

Para cada inserci�n abortada se escribe en el log el evento
CLR\_INSERT (seg�n lo especificado por ARIES). Este muestra la
transacci�n a la que pertenece, los UndoNextLSN, y al final el
registro que se esta removiendo. Los eventos CLR\_INSERT\_INDEX no son
mostrados en la pantalla del log, para mejor lectura.\\

Y por �ltimo se muestra el evento END de finalizaci�n de la
transacci�n 6.\\

