
\newpage
\section{Trabajo futuro}

En este trabajo construimos un motor de base de datos relacional con prop�sitos acad�micos
Que utiliza el m�todo de ARIES para el manejo de transacciones y el aspecto de la recuperaci�n ante ca�das imprevistas del sistema.\\

Esta implementaci�n no contempla algunas temas que son muy interesantes en la actualidad como las transacciones distribuidas y transacciones de larga duraci�n; cuestiones abiertas y de investigaci�n en muchas universidades.\\

Una de las principales ideas a futuro es extender este trabajo implementando una versi�n distribuida de ARIES ( como por ejemplo, D-ARIES ) y que preserve todas las ventajas de ARIES como partial rollbacks , repeatable history, etc.\\

Otro aspecto a mejorar podr�a ser la etapa de An�lisis que se hace durante la recuperaci�n. El cambio consistir�a en paralelizar algunas de las etapas de recuperaci�n de ARIES para ganar eficiencia, otra forma de optimizar ser�a hacer que en la etapa de Redo, �stos sean selectivos, es decir evitar rehacer transacciones que luego se van a deshacer en la etapa de Undo. De esta manera se lograr�a ganar eficiencia en la parte de recuperaci�n.\\

Mientras se hacen las fases de recuperaci�n, que vaya escuchando nuevos pedidos para que sea m�s eficiente el arranque.\\

Soportar media recovery (soporte para copias de seguridad y recupera-ci�n frente a errores ocurridos si se rompe el disco).\\

Deadlock detection en vez de deadlock prevention. Detalles sobre deadlock prevention se pueden encontrar en la secci�n 19.2.2 de \cite{RAMA03}.\\

Agregar un conector est�ndar JDBC (acceder a la DBMS con un cliente JDBC).\\

Mejorar el sistema de �ndices en base a la mejoras descrip-tas en \cite{ARIESIM}, \cite{ARIESKVL} y \cite{ARIESLHS}.\\

Agregar soporte para suspender una transacci�n y luego reanudarla y seguir ejecut�ndola.\\

   
