\title{
	\mbox{\Huge Teaching Databases}\\
	\mbox{}\\
	\mbox{\it An academic implementation of the}\\
	\mbox{\it ARIES family of algorithms}\\
	\mbox{}\\
	Department of Computer Sciences \\
	Facultad de Ciencias Exactas y Naturales\\
	Universidad de Buenos Aires
\date{}
}
\author{}

\maketitle

\begin{center}
	\begin{tabular}{cc}
		Luciano Leggieri&Julian Berl\'in\\
		{\small \verb"lleggieri@dc.uba.ar"}&{\small \verb"jberlin@dc.uba.ar"}\\
		\cr
		Victor Cab\'as&Facundo Pippia\\
		{\small \verb"vcabas@dc.uba.ar"}&{\small \verb"facundomensajes@gmail.com"}\\
	\end{tabular}
	\vskip 5 mm
	Director:\\
	Alejandro Eidelsztein\\
	{\small \verb"ae0n@dc.uba.ar"}\\
\end{center}

{\centering {\bf Abstract}\\}
ARIES is a popular family of recovery algorithms which focuses on a steal / no-force approach. The purpose of this project is to provide an academic but fully functional database engine. It also shows the internal dynamics of those components responsible for the engine's transactional behavior.
This system follows the Client - Server protocol, where all the database complexity is located in the server.

\vskip 5 mm
\begin{center}
	\texttt{Keywords: \\Relational DBMS, ARIES \\Transactions, Recovery Manager,\\Educational technology}
\end{center}

